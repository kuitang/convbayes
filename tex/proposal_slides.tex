%\documentclass[aspectratio=34]{beamer}
\documentclass{beamer}

% Remove the gratuituous footer
\setbeamertemplate{footline}{}
\setbeamertemplate{navigation symbols}{}
%\renewcommand{\insertnavigation}[1]{}

\usepackage{rotating}
\usepackage{subfigure}
\usepackage{algorithm}
\usepackage{algorithmicx}
\usepackage{algpseudocode}
\usepackage{xcolor}

\usepackage{graphicx}
\DeclareMathOperator{\conv}{conv}
\DeclareMathOperator{\st}{s.t.}
\DeclareMathOperator{\dom}{dom}
\DeclareMathOperator{\im}{im}
\DeclareMathOperator{\Ne}{Ne}
\DeclareMathOperator{\sign}{sign}
\DeclareMathOperator{\Var}{Var}
\DeclareMathOperator{\diag}{diag}
\DeclareMathOperator{\vvec}{vec}


%\usepackage{beamerthemesplit}

% Make footnotes visible. Stolen from http://tex.stackexchange.com/questions/5852/beamer-footnote-text-collides-with-navigation-symbols
\addtobeamertemplate{footnote}{\vspace{-6pt}\advance\hsize-0.5cm}{\vspace{6pt}}
\makeatletter
% Alternative A: footnote rule
\renewcommand*{\footnoterule}{\kern -3pt \hrule \@width 2in \kern 8.6pt}
% Alternative B: no footnote rule
% \renewcommand*{\footnoterule}{\kern 6pt}
\makeatother

\setbeamertemplate{bibliography item}{}
\usepackage[style=authoryear-comp,firstinits,doi=true,isbn=false,url=false,eprint=false,backend=biber]{biblatex}
%\usepackage[sorting=none,firstinits,doi=false,isbn=false,url=false,backend=biber]{biblatex}
\bibliography{/Users/kuitang/Dropbox/Library/ML}
% Don't display series
\AtEveryBibitem{\clearlist{series}}
\AtEveryBibitem{\clearfield{series}}
\DeclareSourcemap{
  \maps[datatype=bibtex]{
    \map{
       \step[fieldset=series, null]
    }
  }
}

\DeclareFieldFormat{titlecase}{\MakeTitleCase{#1}}

% Correct casing for journal titles.
\newrobustcmd{\MakeTitleCase}[1]{%
  \ifthenelse{\ifcurrentfield{booktitle}\OR\ifcurrentfield{booksubtitle}%
    \OR\ifcurrentfield{maintitle}\OR\ifcurrentfield{mainsubtitle}%
    \OR\ifcurrentfield{journaltitle}\OR\ifcurrentfield{journalsubtitle}%
    \OR\ifcurrentfield{issuetitle}\OR\ifcurrentfield{issuesubtitle}%
    \OR\ifentrytype{book}\OR\ifentrytype{mvbook}\OR\ifentrytype{bookinbook}%
    \OR\ifentrytype{booklet}\OR\ifentrytype{suppbook}%
    \OR\ifentrytype{collection}\OR\ifentrytype{mvcollection}%
    \OR\ifentrytype{suppcollection}\OR\ifentrytype{manual}%
    \OR\ifentrytype{periodical}\OR\ifentrytype{suppperiodical}%
    \OR\ifentrytype{proceedings}\OR\ifentrytype{mvproceedings}%
    \OR\ifentrytype{reference}\OR\ifentrytype{mvreference}%
    \OR\ifentrytype{report}\OR\ifentrytype{thesis}}
    {#1}
{\MakeSentenceCase{#1}}}

\begin{document}
\begin{frame}
  \frametitle{Objective Functions}
  Regularized log-likelihood
  \begin{align*}
  \max_\theta \ell(\theta;X^{(1:M)},Y^{(1:M)}) &= \sum_{m=1}^M \Big[\langle\phi(X^{(m)},Y^{(m)}),\theta\rangle\\
  &\hspace{1cm} - \log Z(X^{(m)};\theta)\Big] - \frac{\lambda}{2}\|\theta\|^2
  \end{align*}
  But $Z$ is a sum over the exponentially many configurations of $Y^{(m)}$. Replace $\log Z$ with the Bethe approximation, it can be shown this is equivalent to
  \begin{equation*}
    \min_{\tau^{(1:M)}\in\mathcal{T}^M} \frac{1}{2\lambda}\|\theta^*(\tau^{(1:M)})\|^2 - \sum_m H_\rho(\tau^{(m)})
  \end{equation*}
  with $\theta^{*}$ a linear function of $\tau$ and $X, Y$.
\end{frame}

\begin{frame}
  \frametitle{Algorithm}
  \begin{algorithm}[H]
      \caption{Frank-Wolfe Algorithm for MLE}
    \begin{algorithmic}
      \State Let $t := 0$, $\tau$ be the uniform distribution, $h(\tau)$ the objective function of the previous slide.
      \Repeat
        \State Compute gradient $g_t := \nabla h(\tau_t)$
        \For{$m := 1:M$}
          \State Find the MAP assignment for the $m$th example $$\tau_{t}^{*(m)}:=\min_{\tau^{(m)}\in{\cal M}}\left\langle g^{(m)}_t,\tau^{(m)}\right\rangle$$
        \EndFor
        \State Linesearch: Find $\eta^{*}$ to minimize $h((1 - \eta)\tau_t + \eta\tau_{t}^{*})$
        \State Update $\tau_{t+1} := (1 - \eta^{*})\tau_t + \eta^{*}\tau_{t}^{*}$
      \Until{duality gap $< \epsilon$} \Comment{Gap converges at rate $O(1/t)$}
    \end{algorithmic}
  \end{algorithm}
  Inner loop can be parallel, or block-coordinate (randomly pick one sample).
\end{frame}

\begin{frame}
  \frametitle{Results on Linear CRFs (Test Error)}
  \includegraphics[width=\textwidth]{../fig/horseErrCurve200.eps}
\end{frame}

\begin{frame}
  \frametitle{Results on Linear CRFs (Objective Value)}
  \includegraphics[width=\textwidth]{../fig/horseObjCurve200.eps}
\end{frame}

\end{document}

